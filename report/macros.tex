%!TEX root=./main.tex
\let\euscr\mathscr
\graphicspath{ {images/} }

\pgfplotsset{width=10cm,compat=1.9}

\theoremstyle{definition}
\newtheorem{definition}{Definition}[section]
\newtheorem{theorem}{Theorem}[section]
\newtheorem{corollary}{Corollary}[section]
\newtheorem{lemma}{Lemma}[section]
\newcommand{\problemStatement}[3]{%
  \begin{center}
  \begin{tabularx}{\columnwidth}{@{}lX@{}}
  \toprule
  \multicolumn{2}{@{}c@{}}{\textsc{#1}}\tabularnewline
  \midrule
  \bfseries INSTANCE: & #2 \\
  \bfseries QUESTION: & #3 \\
  \bottomrule
  \end{tabularx}
  \end{center}
}
\newtheorem{problem}{Problem}
\newtheorem*{problem*}{Problem}

\newcommand{\set}[1]{\{#1\}}
\newcommand{\tuple}[1]{\langle#1\rangle}
\newcommand{\nats}{\mathbb{N}}
\newcommand{\rats}{\mathbb{Q}}
\newcommand{\true}{{\sf true}}
\newcommand{\false}{{\sf false}}

\newcommand{\vals}{\euscr{V}}
\newcommand{\methods}{\euscr{M}}
\newcommand{\Times}{\euscr{T}}
\newcommand{\methodOf}[1]{{\sf m}(#1)}
\newcommand{\procOf}[1]{{\sf p}(#1)}
\newcommand{\valOf}[1]{{\sf v}(#1)}
\newcommand{\argOf}[1]{{\sf arg}(#1)}
\newcommand{\returnOf}[1]{{\sf ret}(#1)}
\newcommand{\invTimeOf}[1]{{\sf inv}(#1)}
\newcommand{\resTimeOf}[1]{{\sf res}(#1)}
\newcommand{\SetSpec}{T_{\tt set}}
\newcommand{\StackSpec}{T_{\tt stack}}
\newcommand{\QueueSpec}{T_{\tt queue}}
\newcommand{\PQueueSpec}{T_{\tt p.queue}}
\newcommand{\DequeSpec}{T_{\tt deque}}

\newcommand{\enq}{{\tt enq}}
\newcommand{\deq}{{\tt deq}}
\newcommand{\peek}{{\tt peek}}
\newcommand{\push}{{\tt push}}
\newcommand{\pop}{{\tt pop}}
\newcommand{\minsert}{{\tt add}}
\newcommand{\poll}{{\tt poll}}
\newcommand{\contains}{{\tt contains}}
\newcommand{\remove}{{\tt remove}}
\newcommand{\peekfront}{{\tt peek_{front}}}
\newcommand{\pushfront}{{\tt push_{front}}}
\newcommand{\popfront}{{\tt pop_{front}}}
\newcommand{\peekback}{{\tt peek_{back}}}
\newcommand{\pushback}{{\tt push_{back}}}
\newcommand{\popback}{{\tt pop_{back}}}

\newcommand{\perm}[2]{{\sf permissive}_#2(#1)}
\newcommand{\getperm}{{\tt getPermissive}}
\newcommand{\getoverlap}{{\tt getOverlapping}}
\newcommand{\rmsub}{{\tt removeSubhistory}}
\newcommand{\init}{{\tt initialize}}

%%%%%% Macros for algo %%%%%%%%
\newcommand{\seclabel}[1]{\label{sec:#1}}
\newcommand{\secref}[1]{Section~\ref{sec:#1}}
\newcommand{\algolabel}[1]{\label{algo:#1}}
\newcommand{\algoref}[1]{Algorithm~\ref{algo:#1}}
\SetKwProg{myfun}{function}{}{}
\SetKwProg{myproc}{procedure}{}{}
\SetKwProg{myhandler}{handler}{}{}
\SetKwFunction{EmptyCheck}{EmptyCheck}
\SetKwFunction{SetLin}{SetLin}
\SetKwFunction{StackLin}{StackLin}
\SetKwFunction{DequeLin}{DequeLin}
\SetKwFunction{QueueLin}{QueueLin}
\SetKwFunction{PQLin}{PQLin}
\SetKwFunction{MPV}{MaxPermVals}
\SetKwInput{Input}{Input}
\SetKwInOut{Output}{Output}
\SetKw{Let}{let}
\SetKw{Break}{break}
\SetKw{Continue}{continue}
\SetKw{AND}{and}
\SetKw{OR}{or}
\SetKw{NOT}{not}
\SetKw{declare}{declare}
\SetKw{report}{report}
\SetKw{exit}{exit}
\SetKwFor{Foreach}{for each}{}{}%
\SetKwFor{RepTimes}{repeat}{times}{end}
\DontPrintSemicolon